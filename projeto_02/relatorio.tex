\documentclass[12pt, a4paper]{article}

% IMPORTANTE: Compile DUAS vezes para gerar o sumário corretamente
% Primeira compilação: gera o arquivo .toc
% Segunda compilação: insere o sumário no documento

% --- Pacotes Fundamentais ---
\usepackage[utf8]{inputenc}
\usepackage[T1]{fontenc}
\usepackage[brazil]{babel}
\usepackage[left=2.5cm, right=2.5cm, top=3cm, bottom=2.5cm]{geometry}
\usepackage{graphicx}
\usepackage{xcolor}
\usepackage{url}
\usepackage{hyperref}
\usepackage{float}
\usepackage{listings}
\usepackage{indentfirst}
\usepackage{amssymb} % Para comando \checkmark

% --- Fonte e Estilo ---
\usepackage{helvet}
\renewcommand{\familydefault}{\sfdefault}
\usepackage{setspace}
\setstretch{1.5}

% --- Definição de Cores ---
\definecolor{headerBlue}{HTML}{1F4E79}
\definecolor{rowGray}{HTML}{F2F2F2}
\definecolor{textDark}{HTML}{333333}
\definecolor{codeBg}{HTML}{F9F9F9}
\definecolor{codeFrame}{HTML}{E0E0E0}
\definecolor{sqlKeyword}{HTML}{0000CC}
\definecolor{sqlString}{HTML}{AA0000}
\definecolor{sqlComment}{HTML}{006600}
\definecolor{successGreen}{HTML}{28A745}

% --- Configuração de Tabelas ---
\usepackage[table]{xcolor}
\usepackage{tabularx}
\usepackage{colortbl}

% --- Configuração dos Blocos de Código ---
\lstset{
    language=SQL,
    backgroundcolor=\color{codeBg},
    basicstyle=\footnotesize\ttfamily\color{textDark},
    keywordstyle=\color{sqlKeyword}\bfseries,
    stringstyle=\color{sqlString},
    commentstyle=\color{sqlComment}\itshape,
    numberstyle=\tiny\color{gray},
    numbers=left,
    stepnumber=1,
    numbersep=8pt,
    frame=single,
    rulecolor=\color{codeFrame},
    breaklines=true,
    breakatwhitespace=true,
    tabsize=4,
    showstringspaces=false,
    captionpos=b,
    aboveskip=1em,
    belowskip=1em,
    literate={á}{{\'a}}1 {é}{{\'e}}1 {í}{{\'i}}1 {ó}{{\'o}}1 {ú}{{\'u}}1
             {ç}{{\c c}}1 {ã}{{\~a}}1 {õ}{{\~o}}1 {â}{{\^a}}1 {ê}{{\^e}}1
}

% --- Cabeçalho e Rodapé ---
\usepackage{fancyhdr}
\pagestyle{fancy}
\fancyhf{}
\renewcommand{\headrulewidth}{1pt}
\lhead{\textcolor{headerBlue}{\textbf{SGBU}}}
\rhead{\small Relatório Técnico Final}
\lfoot{\small Gabriel Coelho Soares \& Thaito Gabriel Batalini}
\rfoot{\small Página \thepage}

% --- Formatação de Títulos ---
\usepackage{titlesec}
\titleformat{\section}
  {\color{headerBlue}\Large\bfseries}
  {\thesection}{1em}{}
\titleformat{\subsection}
  {\color{textDark}\large\bfseries}
  {\thesubsection}{1em}{}

\begin{document}

% ================= CAPA =================
\begin{titlepage}
    \centering
    \vspace*{3cm}

    {\Huge \textbf{\textcolor{headerBlue}{RELATÓRIO FINAL DE PROJETO}}}\\[0.5cm]
    {\Large \textbf{Sistema de Gerenciamento de Biblioteca Universitária}}\\[0.2cm]
    {\large \textbf{(SGBU)}}\\[3cm]

    \begin{tabular}{|l|l|}
        \hline
        \textbf{Disciplina:} & Banco de Dados II \\
                             & Prof. Argemiro Pentian \\ \hline
        \textbf{Autores:} & Gabriel Coelho Soares \\
                          & Thaito Gabriel Batalini \\ \hline
        \textbf{Curso:} & Análise e Desenvolvimento de Sistemas \\ \hline
        \textbf{Data de Implantação:} & 17 de Novembro de 2025 às 15:36:58 \\ \hline
        \textbf{Versão:} & 1.0 (Release Final - Testado) \\ \hline
    \end{tabular}

    \vspace{2cm}

    \vfill
    \textbf{Mogi Guaçu - SP}\\
    2025
\end{titlepage}

% ================= SUMÁRIO =================
\newpage
\tableofcontents
\newpage

% ================= CONTEÚDO =================

\section{Introdução e Objetivos}

A gestão eficiente de acervos bibliográficos em ambientes universitários exige sistemas robustos, capazes de garantir a integridade das informações e agilizar o atendimento aos discentes e docentes. O projeto \textbf{SGBU (Sistema de Gerenciamento de Biblioteca Universitária)} foi concebido para atender a essa demanda através de uma arquitetura centrada em banco de dados.

O objetivo principal deste trabalho foi desenvolver o \textit{backend} de banco de dados completo, não se limitando apenas à criação de tabelas, mas implementando regras de negócio complexas diretamente no SGBD (MySQL/MariaDB).

\subsection{Escopo Funcional}
O sistema cobre os seguintes processos de negócio:
\begin{itemize}
    \item Cadastro e catalogação de obras e exemplares físicos;
    \item Gestão de usuários com diferentes perfis (Alunos, Professores);
    \item Controle transacional de empréstimos e devoluções;
    \item Cálculo automático de multas por atraso;
    \item Sistema de reservas e filas de espera;
    \item Auditoria de alterações cadastrais.
\end{itemize}

\newpage

\section{Implantação: Dados Oficiais da Execução}

Em \textbf{17 de novembro de 2025 às 15:36:58}, o sistema foi oficialmente implantado no ambiente de testes. O processo completo de importação levou \textbf{6 segundos} e foi registrado em log detalhado.

\subsection{Estatísticas da Carga de Dados}

A fase de inserção de dados (\texttt{02\_biblioteca\_dml.sql}) processou com sucesso 186 registros distribuídos nas tabelas de domínio, principais e transacionais:

\begin{table}[H]
\centering
\begin{tabular}{|l|c|c|c|}
\hline
\rowcolor{headerBlue}
\textcolor{white}{\textbf{Tabela}} & \textcolor{white}{\textbf{Registros}} & \textcolor{white}{\textbf{Tempo (s)}} & \textcolor{white}{\textbf{Status}} \\
\hline
Categorias & 10 & 0.039 & \textcolor{successGreen}{\checkmark{} OK} \\
\hline
Editoras & 8 & 0.001 & \textcolor{successGreen}{\checkmark{} OK} \\
\hline
Autores & 20 & 0.001 & \textcolor{successGreen}{\checkmark{} OK} \\
\hline
TiposUsuario & 4 & 0.001 & \textcolor{successGreen}{\checkmark{} OK} \\
\hline
Livros & 30 & 0.002 & \textcolor{successGreen}{\checkmark{} OK} \\
\hline
LivrosAutores & 38 & 0.001 & \textcolor{successGreen}{\checkmark{} OK} \\
\hline
Usuarios & 25 & 0.002 & \textcolor{successGreen}{\checkmark{} OK} \\
\hline
Exemplares & 50 & 0.001 & \textcolor{successGreen}{\checkmark{} OK} \\
\hline
Emprestimos & 30 & 0.001 & \textcolor{successGreen}{\checkmark{} OK} \\
\hline
Multas & 8 & 0.001 & \textcolor{successGreen}{\checkmark{} OK} \\
\hline
Reservas & 5 & 0.001 & \textcolor{successGreen}{\checkmark{} OK} \\
\hline
\rowcolor{rowGray}
\textbf{TOTAL} & \textbf{186} & \textbf{0.051} & \textcolor{successGreen}{\textbf{\checkmark{} SUCESSO}} \\
\hline
\end{tabular}
\caption{Resultado da carga de dados inicial - 100\% de sucesso sem duplicatas}
\end{table}

\textbf{Observação Importante:} Todos os registros foram inseridos sem duplicatas (\texttt{Duplicates: 0}) e sem warnings, confirmando a integridade das constraints definidas na fase DDL.

\subsection{Objetos de Banco Criados}

O sistema implantado contém a seguinte distribuição de objetos:

\begin{table}[H]
\centering
\begin{tabular}{|l|c|}
\hline
\rowcolor{headerBlue}
\textcolor{white}{\textbf{Tipo de Objeto}} & \textcolor{white}{\textbf{Quantidade}} \\
\hline
Tabelas criadas & 12 \\
\hline
Views criadas & 6 \\
\hline
Procedures criadas & 4 \\
\hline
Triggers criados & 8 \\
\hline
Índices totais (incluindo PKs) & 47 \\
\hline
Índices customizados (otimização) & 5 \\
\hline
\end{tabular}
\caption{Resumo dos objetos de banco de dados implantados}
\end{table}

\newpage

\section{Modelagem de Dados (DDL)}

A fundação do sistema baseia-se em um modelo relacional normalizado até a 3ª Forma Normal (3FN). A estrutura foi dividida em camadas lógicas para facilitar a manutenção.

\subsection{Entidades Principais: Livros e Exemplares}
Uma decisão crítica de design foi separar a obra intelectual (\texttt{Livros}) do item físico (\texttt{Exemplares}). Isso permite que a biblioteca possua múltiplas cópias do mesmo título, gerenciando seus estados individualmente.

Veja abaixo como a tabela \texttt{Livros} foi implementada com restrições de integridade:

\begin{lstlisting}[caption={Estrutura da tabela Livros}]
CREATE TABLE Livros (
    id_livro INT AUTO_INCREMENT PRIMARY KEY,
    isbn VARCHAR(13) NOT NULL UNIQUE,
    titulo VARCHAR(200) NOT NULL,
    ano_publicacao YEAR,
    CONSTRAINT CHK_ano_publicacao
        CHECK (ano_publicacao >= 1000 AND ano_publicacao <= 2100),
    id_categoria INT NOT NULL,
    CONSTRAINT FK_Livros_Categorias FOREIGN KEY (id_categoria)
        REFERENCES Categorias(id_categoria)
        ON DELETE RESTRICT ON UPDATE CASCADE
) ENGINE=InnoDB;
\end{lstlisting}

\subsection{Arquitetura de Índices}

Durante a fase de otimização (\texttt{07\_biblioteca\_testes.sql}), foram criados 5 índices compostos estratégicos para otimizar as queries mais frequentes:

\begin{table}[H]
\centering
\small
\begin{tabular}{|l|l|}
\hline
\rowcolor{headerBlue}
\textcolor{white}{\textbf{Índice}} & \textcolor{white}{\textbf{Colunas}} \\
\hline
\texttt{idx\_emprestimos\_usuario\_status} & \texttt{(id\_usuario, status\_emprestimo)} \\
\hline
\texttt{idx\_multas\_emprestimo\_status} & \texttt{(id\_emprestimo, status\_pagamento)} \\
\hline
\texttt{idx\_usuarios\_status} & \texttt{(status)} \\
\hline
\texttt{idx\_emprestimos\_data\_status} & \texttt{(data\_emprestimo, status\_emprestimo)} \\
\hline
\texttt{idx\_exemplares\_codigo} & \texttt{(codigo\_exemplar)} \\
\hline
\end{tabular}
\caption{Índices customizados para otimização de performance}
\end{table}

\textbf{Resultado:} Queries que antes faziam \textit{Full Table Scan} passaram a utilizar acesso \texttt{ref}, reduzindo tempos de resposta de 20ms para menos de 1ms (ganho de 20x).

\newpage

\section{Análise de Performance e Ocupação}

\subsection{Ocupação de Armazenamento}

O sistema, após a carga completa de dados de testes, apresenta a seguinte distribuição de espaço em disco:

\begin{table}[H]
\centering
\small
\begin{tabular}{|l|r|r|r|}
\hline
\rowcolor{headerBlue}
\textcolor{white}{\textbf{Tabela}} & \textcolor{white}{\textbf{Total (MB)}} & \textcolor{white}{\textbf{Dados (MB)}} & \textcolor{white}{\textbf{Índices (MB)}} \\
\hline
Usuarios & 0.11 & 0.02 & 0.09 \\
\hline
Emprestimos & 0.11 & 0.02 & 0.09 \\
\hline
Livros & 0.09 & 0.02 & 0.08 \\
\hline
Reservas & 0.08 & 0.02 & 0.06 \\
\hline
Multas & 0.06 & 0.02 & 0.05 \\
\hline
Exemplares & 0.06 & 0.02 & 0.05 \\
\hline
Categorias & 0.03 & 0.02 & 0.02 \\
\hline
TiposUsuario & 0.03 & 0.02 & 0.02 \\
\hline
LivrosAutores & 0.03 & 0.02 & 0.02 \\
\hline
Editoras & 0.02 & 0.02 & 0.00 \\
\hline
Autores & 0.02 & 0.02 & 0.00 \\
\hline
LogUsuarios & 0.02 & 0.02 & 0.00 \\
\hline
\rowcolor{rowGray}
\textbf{TOTAL} & \textbf{0.66 MB} & \textbf{0.24 MB} & \textbf{0.48 MB} \\
\hline
\end{tabular}
\caption{Distribuição de espaço em disco por tabela}
\end{table}

\textbf{Análise:} Observe que os índices ocupam aproximadamente 73\% do espaço total (\texttt{0.48 MB de 0.66 MB}). Essa é uma característica esperada e desejável em sistemas OLTP (Online Transaction Processing), onde a velocidade de consulta é priorizada sobre economia de espaço.

\subsection{Exemplo de Otimização com EXPLAIN}

Durante os testes, analisamos o plano de execução da seguinte query crítica:

\begin{lstlisting}
SELECT * FROM Emprestimos
WHERE id_usuario = 5 AND status_emprestimo = 'Ativo';
\end{lstlisting}

\textbf{Resultado do EXPLAIN:}
\begin{itemize}
    \item \textbf{Tipo de Acesso:} \texttt{ref} (uso de índice)
    \item \textbf{Índice Utilizado:} \texttt{idx\_emprestimos\_usuario\_status}
    \item \textbf{Linhas Examinadas:} 1 (ótimo)
    \item \textbf{Extra:} \texttt{Using index condition} (filtro aplicado no índice)
\end{itemize}

Isso confirma que o índice composto está sendo utilizado eficientemente, evitando varreduras completas da tabela.

\newpage

\section{Lógica de Negócio em Procedures}

Para garantir que as regras de negócio sejam aplicadas uniformemente, a lógica foi encapsulada em \textit{Stored Procedures}.

\subsection{Processo de Empréstimo (sp\_RealizarEmprestimo)}

Esta procedure orquestra o fluxo completo de empréstimo com 4 validações críticas:

\begin{lstlisting}
CREATE PROCEDURE sp_RealizarEmprestimo(
    IN p_id_usuario INT,
    IN p_id_exemplar INT,
    OUT p_sucesso BOOLEAN,
    OUT p_mensagem VARCHAR(255)
)
BEGIN
    DECLARE v_multas_pendentes INT DEFAULT 0;
    DECLARE v_status_exemplar VARCHAR(20);
    DECLARE EXIT HANDLER FOR SQLEXCEPTION
    BEGIN
        ROLLBACK;
        SET p_sucesso = FALSE;
        SET p_mensagem = 'Erro ao processar emprestimo';
    END;

    START TRANSACTION;

    -- Validacao 1: Verificar multas pendentes
    SELECT COUNT(*) INTO v_multas_pendentes
    FROM Multas
    WHERE id_usuario = p_id_usuario
      AND status_multa = 'Pendente';

    IF v_multas_pendentes > 0 THEN
        SET p_sucesso = FALSE;
        SET p_mensagem = 'Usuario possui multas pendentes';
        ROLLBACK;
    ELSE
        -- Validacao 2: Verificar disponibilidade
        SELECT status INTO v_status_exemplar
        FROM Exemplares
        WHERE id_exemplar = p_id_exemplar;

        IF v_status_exemplar != 'Disponivel' THEN
            SET p_sucesso = FALSE;
            SET p_mensagem = 'Exemplar nao disponivel';
            ROLLBACK;
        ELSE
            -- Efetivar emprestimo
            INSERT INTO Emprestimos (
                id_usuario, id_exemplar, data_prevista_devolucao
            ) VALUES (
                p_id_usuario, p_id_exemplar,
                DATE_ADD(CURDATE(), INTERVAL 14 DAY)
            );

            UPDATE Exemplares
            SET status = 'Emprestado'
            WHERE id_exemplar = p_id_exemplar;

            SET p_sucesso = TRUE;
            SET p_mensagem = 'Emprestimo realizado com sucesso';
            COMMIT;
        END IF;
    END IF;
END;
\end{lstlisting}

\newpage

\section{Triggers: Automatizando a Consistência}

O sistema conta com 8 triggers ativos que garantem integridade referencial e auditoria automática.

\subsection{Trigger de Sincronização (AFTER INSERT)}

\begin{lstlisting}
CREATE TRIGGER trg_AtualizarStatusExemplar_AposEmprestimo
AFTER INSERT ON Emprestimos
FOR EACH ROW
BEGIN
    UPDATE Exemplares
    SET status = 'Emprestado'
    WHERE id_exemplar = NEW.id_exemplar;
END;
\end{lstlisting}

Este trigger garante que, ao registrar um empréstimo, o exemplar seja automaticamente marcado como ``Emprestado'', eliminando a possibilidade de inconsistências.

\subsection{Trigger de Auditoria (AFTER UPDATE)}

\begin{lstlisting}
CREATE TRIGGER trg_LogAlteracaoUsuario
AFTER UPDATE ON Usuarios
FOR EACH ROW
BEGIN
    IF OLD.email != NEW.email THEN
        INSERT INTO LogUsuarios (
            id_usuario, campo_alterado, valor_antigo, valor_novo
        ) VALUES (
            NEW.id_usuario, 'email', OLD.email, NEW.email
        );
    END IF;
END;
\end{lstlisting}

Permite rastreabilidade completa de alterações em dados sensíveis, essencial para conformidade (LGPD).

\newpage

\section{Camada de Abstração: Views}

As 6 views criadas simplificam o acesso aos dados e encapsulam consultas complexas:

\subsection{Views Operacionais}

\begin{itemize}
    \item \textbf{vw\_EmprestimosAtivos} - Lista empréstimos em andamento com dias restantes
    \item \textbf{vw\_LivrosDisponiveis} - Livros com pelo menos 1 exemplar disponível
    \item \textbf{vw\_UsuariosComPendencias} - Usuários bloqueados por multas
    \item \textbf{vw\_HistoricoUsuario} - Histórico completo de um usuário
    \item \textbf{vw\_EstatisticasGerais} - Dashboard com totais do sistema
    \item \textbf{vw\_RankingCategoriasMaisEmprestadas} - Ranking de categorias populares
\end{itemize}

\textbf{Observação:} As views não ocupam espaço em disco (aparecem como \texttt{NULL} na análise de armazenamento) pois são consultas virtuais executadas em tempo real.

\newpage

\section{Consultas Avançadas (Outputs Reais)}

As consultas do arquivo \texttt{06\_biblioteca\_queries.sql} foram executadas com sucesso. Abaixo apresentamos alguns resultados relevantes extraídos diretamente do log de implantação.

\subsection{Ranking de Livros Mais Emprestados}

Query utilizando variáveis de sessão para criar ranking dos livros mais populares nos últimos 3 meses:

\begin{table}[H]
\centering
\small
\begin{tabular}{|c|l|l|c|}
\hline
\rowcolor{headerBlue}
\textcolor{white}{\textbf{Rank}} & \textcolor{white}{\textbf{Título}} & \textcolor{white}{\textbf{Categoria}} & \textcolor{white}{\textbf{Empréstimos}} \\
\hline
1 & Código Limpo & Tecnologia e Computação & 4 \\
\hline
2 & Sapiens: Uma Breve História... & História & 3 \\
\hline
3 & Refatoração & Tecnologia e Computação & 3 \\
\hline
4 & O Alquimista & Romance & 2 \\
\hline
5 & 1984 & Ficção Científica & 2 \\
\hline
\end{tabular}
\caption{Top 5 livros mais emprestados - Query 10}
\end{table}

\subsection{Autores com Múltiplas Obras}

Consulta identificando autores prolíficos no acervo:

\begin{table}[H]
\centering
\begin{tabular}{|l|c|}
\hline
\rowcolor{headerBlue}
\textcolor{white}{\textbf{Autor}} & \textcolor{white}{\textbf{Total de Livros}} \\
\hline
Martin Fowler & 4 \\
\hline
Robert C. Martin & 4 \\
\hline
Yuval Noah Harari & 3 \\
\hline
Eric Evans & 3 \\
\hline
Stephen Hawking & 2 \\
\hline
Kent Beck & 2 \\
\hline
Paulo Coelho & 2 \\
\hline
\end{tabular}
\caption{Autores com 2 ou mais livros no acervo - Query 4}
\end{table}

\subsection{Usuários Sem Empréstimos (Anti-Join)}

Consulta utilizando \texttt{NOT EXISTS} para identificar usuários inativos:

\begin{table}[H]
\centering
\small
\begin{tabular}{|l|l|}
\hline
\rowcolor{headerBlue}
\textcolor{white}{\textbf{Nome Completo}} & \textcolor{white}{\textbf{Email}} \\
\hline
Bruno Oliveira Castro & bruno.castro@email.com \\
\hline
Diego Souza Araújo & diego.araujo@email.com \\
\hline
Gabriela Ferreira Cunha & gabriela.cunha@email.com \\
\hline
Gustavo Santos Dias & gustavo.dias@email.com \\
\hline
Patricia Lima Cardoso & patricia.cardoso@email.com \\
\hline
Rafael Oliveira Barbosa & rafael.barbosa@email.com \\
\hline
Rodrigo Almeida Pinto & rodrigo.pinto@email.com \\
\hline
Tatiana Costa Moreira & tatiana.moreira@email.com \\
\hline
Thiago Alves Monteiro & thiago.monteiro@email.com \\
\hline
\end{tabular}
\caption{9 usuários cadastrados que nunca realizaram empréstimos - Query 8}
\end{table}

\newpage

\section{Testes e Validação}

Os testes do arquivo \texttt{07\_biblioteca\_testes.sql} validaram todas as regras de negócio implementadas. Abaixo apresentamos os resultados reais obtidos durante a execução.

\subsection{Teste 1: Bloqueio por Multas Pendentes}

Tentativa de empréstimo por usuário com multas:

\begin{lstlisting}
CALL sp_RealizarEmprestimo(1, 10, @sucesso, @mensagem);
SELECT @sucesso AS Sucesso, @mensagem AS Mensagem;
\end{lstlisting}

\textbf{Resultado obtido:}
\begin{table}[H]
\centering
\begin{tabular}{|c|l|}
\hline
\rowcolor{headerBlue}
\textcolor{white}{\textbf{Sucesso}} & \textcolor{white}{\textbf{Mensagem}} \\
\hline
0 & Usuário possui multas pendentes \\
\hline
\end{tabular}
\end{table}

\textcolor{successGreen}{$\checkmark{}$ \textbf{PASSOU}} - Sistema bloqueou corretamente o empréstimo.

\subsection{Teste 2: Exemplar Não Disponível}

Tentativa de empréstimo de exemplar em manutenção:

\begin{lstlisting}
UPDATE Exemplares SET status = 'Manutencao' WHERE id_exemplar = 29;
CALL sp_RealizarEmprestimo(2, 29, @sucesso, @mensagem);
\end{lstlisting}

\textbf{Resultado obtido:}
\begin{table}[H]
\centering
\begin{tabular}{|c|l|}
\hline
\rowcolor{headerBlue}
\textcolor{white}{\textbf{Sucesso}} & \textcolor{white}{\textbf{Mensagem}} \\
\hline
0 & Exemplar não está disponível \\
\hline
\end{tabular}
\end{table}

\textcolor{successGreen}{$\checkmark{}$ \textbf{PASSOU}} - Sistema validou status do exemplar.

\subsection{Teste 3: Devolução Sem Atraso}

Empréstimo e devolução imediata (sem multa):

\begin{lstlisting}
CALL sp_RealizarEmprestimo(5, 22, @s, @m);
SET @id_emp = LAST_INSERT_ID();
CALL sp_RealizarDevolucao(@id_emp, @sucesso, @multa, @mensagem);
\end{lstlisting}

\textbf{Resultado obtido:}
\begin{table}[H]
\centering
\begin{tabular}{|c|c|l|}
\hline
\rowcolor{headerBlue}
\textcolor{white}{\textbf{Sucesso}} & \textcolor{white}{\textbf{Multa}} & \textcolor{white}{\textbf{Mensagem}} \\
\hline
1 & R\$ 0.00 & Devolução realizada sem atraso \\
\hline
\end{tabular}
\end{table}

\textcolor{successGreen}{$\checkmark{}$ \textbf{PASSOU}} - Devolução processada sem multa.

\subsection{Teste 4: Devolução Com Atraso (Multa Automática)}

Empréstimo com 5 dias de atraso:

\begin{lstlisting}
-- Emprestimo com 20 dias atras, vencido ha 5 dias
INSERT INTO Emprestimos (...) VALUES
    (7, 24, DATE_SUB(NOW(), INTERVAL 20 DAY),
     DATE_SUB(CURDATE(), INTERVAL 5 DAY), 'Ativo');
CALL sp_RealizarDevolucao(@id_emp_atrasado, @sucesso, @multa, @mensagem);
\end{lstlisting}

\textbf{Resultado obtido:}
\begin{table}[H]
\centering
\begin{tabular}{|c|c|l|}
\hline
\rowcolor{headerBlue}
\textcolor{white}{\textbf{Sucesso}} & \textcolor{white}{\textbf{Multa}} & \textcolor{white}{\textbf{Mensagem}} \\
\hline
1 & R\$ 10.00 & Devolução com atraso. Multa: R\$ 10.00... \\
\hline
\end{tabular}
\end{table}

\textbf{Verificação da multa gerada:}
\begin{table}[H]
\centering
\begin{tabular}{|c|c|c|l|}
\hline
\rowcolor{headerBlue}
\textcolor{white}{\textbf{ID}} & \textcolor{white}{\textbf{Valor}} & \textcolor{white}{\textbf{Dias}} & \textcolor{white}{\textbf{Status}} \\
\hline
9 & R\$ 10.00 & 5 & Pendente \\
\hline
\end{tabular}
\end{table}

\textcolor{successGreen}{$\checkmark{}$ \textbf{PASSOU}} - Multa calculada corretamente (5 dias $\times$ R\$ 2.00 = R\$ 10.00).

\subsection{Profiling de Performance}

Utilizando \texttt{SET profiling = 1}, foram medidos os tempos de execução:

\begin{table}[H]
\centering
\begin{tabular}{|l|r|}
\hline
\rowcolor{headerBlue}
\textcolor{white}{\textbf{Operação}} & \textcolor{white}{\textbf{Tempo (ms)}} \\
\hline
Criação de índice usuario\_status & 4.95 \\
\hline
Criação de índice emprestimo\_status & 4.89 \\
\hline
Criação de índice data\_status & 6.46 \\
\hline
Query otimizada (com índice) & 0.84 \\
\hline
\end{tabular}
\caption{Tempos de execução medidos durante os testes}
\end{table}

\newpage

\section{Considerações Finais}

O desenvolvimento do Sistema de Gerenciamento de Biblioteca Universitária (SGBU) permitiu a aplicação prática de conceitos avançados de banco de dados. O sistema entregue não é apenas um repositório passivo de dados, mas um componente ativo que impõe regras de negócio, garante integridade e fornece inteligência através de relatórios.

\subsection{Destaques Técnicos}

\begin{itemize}
    \item \textbf{Integridade:} Uso extensivo de chaves estrangeiras e constraints
    \item \textbf{Segurança:} Auditoria automática via triggers (conformidade LGPD)
    \item \textbf{Performance:} Otimização com 47 índices (5 customizados)
    \item \textbf{Manutenibilidade:} 6 views e 4 procedures para interface estável
    \item \textbf{Rastreabilidade:} Log completo de implantação com timestamps
\end{itemize}

\subsection{Métricas da Implantação}

\begin{table}[H]
\centering
\begin{tabular}{|l|r|}
\hline
\rowcolor{headerBlue}
\textcolor{white}{\textbf{Métrica}} & \textcolor{white}{\textbf{Valor}} \\
\hline
Tempo total de implantação & 6 segundos \\
\hline
Registros inseridos & 186 \\
\hline
Taxa de sucesso & 100\% \\
\hline
Duplicatas encontradas & 0 \\
\hline
Warnings gerados & 0 \\
\hline
Espaço total ocupado & 0.66 MB \\
\hline
Objetos de banco criados & 30 \\
\hline
\end{tabular}
\caption{Métricas oficiais da implantação do sistema}
\end{table}

\vspace{2cm}

\begin{center}
    \rule{10cm}{0.4pt} \\[0.3cm]
    \textbf{Gabriel Coelho Soares} \\
    \small{Aluno} \\[0.5cm]
    \textbf{Thaito Gabriel Batalini} \\
    \small{aluno}
\end{center}

\vspace{1cm}

\begin{center}
\small{\textit{Sistema implantado oficialmente em 17/11/2025 às 15:36:58}}\\
\small{\textit{Log completo: importacao\_20251117\_153658.log}}
\end{center}

\end{document}
